 \textsc{\newcommand*{\annot}[1]{\tag*{\footnotesize{\textcolor{black!40}{\(#1\)}}}}
\newcommand*{\mystrut}{\rule(-0.05\baselineskip){0pt}{1.2\baselineskip}}
\newcommand*{\mybox}[1]{\textcolor{black!40}{\framebox{\mystrut #1}}}

% Code to render multiple blank lines for a algebraic derivation. Hints can be written by students of lines on the right.
\newcommand{\derivblank}[1]{%
	\begin{align*}
	\xderivblank{#1}%
	&=  \hspace{0.2cm}  \mybox{\underline{ \hspace{0.5\linewidth}}} \annot{\underline{\hspace{0.3\linewidth}}}
	\end{align*}
}
\newcommand\xderivblank[1]{%
	\ifnum#1=0 \else
	&= \hspace{0.2cm}\textcolor{black!40}{\underline{\mystrut\hspace{0.5\linewidth}}} \annot{\underline{\hspace{0.3\linewidth}}} \\
	\xderivblank{\numexpr#1-1}\fi}

% Code to write multiple lines of equivalent equations.
\newcommand{\eqblank}[1]{%
	\begin{align*}
	\xeqblank{#1}%
	&\Leftrightarrow  \hspace{0.2cm}  \mybox{\underline{ \hspace{0.5\linewidth}}} \annot{\underline{\hspace{0.3\linewidth}}}
	\end{align*}
}
\newcommand\xeqblank[1]{%
	\ifnum#1=0 \else
	&\Leftrightarrow \hspace{0.2cm}\underline{\mystrut\hspace{0.5\linewidth}} \annot{\underline{\hspace{0.3\linewidth}}} \\
	\xeqblank{\numexpr#1-1}\fi}}
